\documentclass[a4paper,12pt]{article}
\usepackage [T2A]{fontenc}
\usepackage[utf8]{inputenc}
\usepackage [english,ukrainian] {babel}
\usepackage{indentfirst}
\usepackage{amsmath}
\usepackage{setspace}
\usepackage[left=3.5cm,top=2cm,right=1cm,bottom=2cm,nohead,nofoot]{geometry}
\usepackage{enumerate}

\begin{document}

\tableofcontents
\addcontentsline{toc}{section}{Вступ}
\addcontentsline{toc}{section}{Постановка задачі}
\addcontentsline{toc}{section}{Висновок}
\addcontentsline{toc}{section}{Література}
\addcontentsline{toc}{section}{Додаток}

\newpage


\onehalfspacing
\large

\section*{Вступ}
Для штучних супутників Землі (ШСЗ) однією з найважливіших є проблема електризації та заняття електричного заряду з поверхні ШСЗ в процесі експлуатації.

Суть проблеми полягає в тому, що ШСЗ, які знаходяться на високих орбітах, насамперед на геостаціонарних, піддаються  нерівномірній електризації швидкими електронами. При цьому ШСЗ в цілому, або окремі частини його поверхні, які знаходяться в тіні сонця, заряджаються до високого від’ємного потенціалу відносно оточуючого космічного простору. \cite{report1}. Через неоднакову освітленість діелектричних ділянок ШС виникає різниця потенціалів, яка веде до електричних пробоїв - вони ведуть до збоїв в роботі радіоелектронних приладів і руйнують поверхню супутника. Ефект електричного зарядження особливо посилюється в період геомагнітних бурь, пов’язаних з підвищеною сонячною активністю. У таких випадках від’ємний потенціал геостаціонарного супутника може досягати значних величин. \cite{sharp} \cite{deforest}

Дана робота присвячена розробці програмного забезпечення для моделювання електризації тіл в космічному просторі, а також обчислення значень електричних потенціалів, що будуть накопичуватися на цих тілах або їх частинах.

Знаючи реальні дані (швидкість електронів, густину їх розподілу, швидкість тіла), проводиться серія випробувань (результати щоразу будуть різними за рахунок використання датчика псевдовипадкових чисел) і отримуються значення зміни потенціалу за проміжок часу.

\section*{Постановка задачі}
\textbf{1-2стр}

\section{Огляд}
\subsection{Фізична модель}
При взаємодії космічного апарату (КА) з плазмою, що оточує його в польоті, виникають різноманітні фізичні явища, специфіка яких залежить як від параметрів плазми, так і від характеристик КА, в першу чергу - від властивостей матеріалів, що знаходяться на його поверхні, і від конфігурації апарату. До таких явищ належать: утворення електричного заряду на поверхні КА, розпилення матеріалів, світіння на поверхні і поблизу неї, збудження коливань у плазмі та деякі інші.

Найбільш значний вплив на функціонування КА може здійснювати утворення заряду на його поверхні. Знак і величина електричного заряду, що утворюється на поверхні КА, залежать від співвідношення інтенсивності процесів, що забезпечують надходження на поверхню і видалення з неї позитивно і негативно заряджених частинок, тобто від співвідношення різних складових сумарного електричного струму, що тече через поверхню КА. Основними складовими цього струму є електронний та іонний струми навколишньої плазми, вторинно-емісійні струми, обумовлені первинними плазмовими струмами, і фотоелектронний струм, що виникає під дією короткохвильового випромінювання Сонця. Додаткові складові можуть створюватися деякими видами бортового устаткування КА: електроракетними двигунами, що випускають при роботі плазмові струмені, електронними та іонними прожекторами, що використовуються в наукових експериментах і т.п.

При електризації КА між його поверхнею і навколишнього плазмою виникає різниця потенціалів. Сталий потенціал поверхні, відлічуваний щодо потенціалу незбуреної плазми, визначається умовою динамічної рівноваги, при якому сумарний струм, що тече через поверхню КА, дорівнює нулю. З енергетичних співвідношень випливає, що рівноважний потенціал залежить від середньої енергії частинок плазми, тобто від її температури: чим вища температура плазми, тим більший потенціал може отримати поверхня тіла. В багатокомпонентній космічній плазмі характерне значення максимального потенціалу визначається енергією заряджених частинок, що превалюють в струмовому балансі.

Реальний КА являє собою складну конструкцію з неоднорідною структурою і великою кількістю діелектричних матеріалів на зовнішній поверхні. У зв'язку з цим потенціали окремих ділянок поверхні і елементів конструкції можуть бути різними через відмінності умов потрапляння потоків первинних частинок на ці ділянки та умов їх освітлення, а також через відмінності емісійних властивостей матеріалів поверхні. Відбувається так зване диференційне заряджання КА, при якому між окремими ділянками непровідної поверхні виникають різниці потенціалів.

Описаний вище процес зарядження КА як єдиного провідного тіла прийнято називати загальним зарядженням. Дане поняття можна застосувати і по відношенню до реального КА, проте в цьому випадку воно відноситься до середнього потенціалу КА, що визначається сукупністю всіх електричних зарядів, які знаходяться на його поверхні і елементах конструкції.

Очевидно, що власне електричне поле зарядженого КА є збурюючим фактором, який необхідно враховувати в багатьох випадках при проведенні вимірювань параметрів космічного середовища за допомогою приладів, встановлених на КА. З цієї точки зору явище електризації КА в космічній плазмі аналізувалося ще в середині 1950-х рр. при розробці наукових приладів для перших штучних супутників Землі. Тоді приймалися до уваги потенціали з характерними величинами від часток до одиниць вольт, що, як ми побачимо далі, характерно для випадку зарядження КА на низьких навколоземних орбітах - в іоносфері.

Проте найбільший вплив на бортове обладнання КА мають електростатичні розряди (ЕСР), які можуть виникати між окремими ділянками поверхні і елементами конструкції диференційно зарядженого КА, а також між його поверхнею і навколишнього плазмою. Локальні струми і електромагнітні випромінювання, породжувані ЕСР, створюють значні перешкоди для роботи бортового обладнання КА.

Як фактор, що має серйозний несприятливий вплив на роботу бортових систем КА, явище електризації стало систематично вивчатися на початку 1970-х рр. при запусках КА на геостаціонарну орбіту (кругова екваторіальна орбіта з висотою ~ 36000 км), де, як з'ясувалося пізніше, параметри плазми такі, що значення потенціалів на КА досягають 10-20 кВ.

Геостаціонарна орбіта (ГСО) примітна тим, що на ній кутова швидкість руху КА дорівнює швидкості обертання Землі. Внаслідок цього КА постійно знаходиться над однією точкою земної поверхні (звідси назва орбіти), забезпечуючи тим самим дуже зручні умови для трансляції через нього радіосигналів. Тому геостаціонарні КА працюють головним чином в космічних системах радіозв'язку та телебачення.

На перших геостаціонарних КА, спроектованих без урахування можливого впливу ефектів електризації, спостерігалася велика кількість неполадок в роботі бортового обладнання: відбувалися мимовільні включення і виключення різних пристроїв, змінювалася орієнтація антен, припинялася подача електроенергії від сонячних батарей і т.д., причому аномалії спостерігалися переважно в нічні та ранні ранкові години. Не відразу вдалося зрозуміти, що всі ці ефекти пов'язані з електризацією КА.

Поступово при статистичному аналізі відмов і збоїв у роботі апаратури КА був виявлений кореляційний зв'язок між спостережуваними аномаліями і появою інтенсивних потоків гарячої плазми в області ГСО. На геостаціонарних КА були встановлені прилади для вимірювання параметрів навколишнього плазми і спеціальні датчики для реєстрації електромагнітних перешкод і вимірювання напруженості електричного поля біля поверхні КА. Дані, отримані за допомогою цих приладів, переконливо підтвердили факт виникнення ЕСР на борту КА при електризації під дією гарячої плазми. При характерних значеннях потенціалів на геостаціонарних КА, вимірюваних одиницями і навіть десятками кіловольт, рівень перешкод, створюваних ЕСР, дуже високий, а в деяких випадках ЕСР можуть призводити до руйнування компонентів апаратури та елементів конструкції.

Вжиті теоретичні і лабораторні дослідження явища електризації КА дозволили зрозуміти основні його закономірності і запропонувати методи зниження впливу ефектів електризації на функціонування бортових систем КА. Однак проблема далеко не вичерпана. Створення нових конструкцій КА, підвищення вимог до їх надійності та тривалості функціонування, оснащення КА новими видами устаткування і високочутливою науковою апаратурою - все це потребує подальшого детального вивчення особливостей електризації КА в різних умовах і вдосконалення методів їх захисту.\cite{novikov}

\subsubsection{Струми частинок плазми на поверхні незарядженого тіла}
На поверхню тіла, внесеного в плазму, надходять потоки електронів та іонів, зумовлені тепловим рухом частинок. При однаковій енергії електронів та іонів, яка визначається  емпературою плазми, електрони мають значно більш високу швидкість в порівнянні з іонами через різницю мас частинок. Тому спочатку, поки внесене в плазму тіло не заряджена, потік електронів, що падає на поверхню, перевищує потік позитивних іонів, і тіло заряджається негативно. Далі надходження заряджених частинок на поверхню відбувається в умовах дії на них електричного поля, яке по відношенню до електронів є гальмуючим, а по відношенню до позитивних іонів - прискорюючим. Це в результаті призводить до рівності потоків електронів та іонів при деякому негативному потенціалі поверхні. Такий простий випадок заряжения тіла в двокомпонентної плазмі розглядається в теорії плазмового зонда, відомого у фізиці як зонд Ленгмюра.


Як уже зазначалося, якщо внесене в плазму тіло не заряджена, тобто знаходиться при потенціалі плазми, струми, що течуть на поверхні, обумовлені тільки тепловим рухом частинок.


В цьому випадку щільність струму частинок плазми одного виду з концентрацією n визначається виразом
\[
j = -en \int (sv) f(v) \mathrm{d}x,
\]
де s -- нормаль до поверхні, що розглядається, а добуток $(sv)$ -- нормальна складова швидкості. Інтегрування ведеться по зовнішній відносно поверхні півсфері.

Для незарядженої поверхні при максвелівському розподілі частинок за швидкостями отримаємо
\[
j_0 = \left( \frac{kT}{2 \pi m} \right)^\frac{1}{2}.
\]
У ізотермічної електронно-протонній плазмі з однаковою концентрацією часток ($T_e = T_p, n_e = n_p$) відношення щільності електронного струму на незарядженій поверхні до щільності протонного струму визначається виразом \cite{novikov}
\[
\frac{j_{e0}}{j_{p0}} = \left( \frac{m_p}{m_e} \right)^\frac{1}{2} = \sqrt{1836} \approx 43.
\]

\subsection{Метод Монте-Карло}
Моделювання проводиться за методом Монте-Карло \cite{sobol}. Цей метод заснований на багатократних реалізація стохастичного процесу і застосовується в тих випадках, коли побудова точної математичної моделі неможлива або значно ускладнена. Прояв методів статистичного моделювання в різних областях прикладної математики, як правило, пов’язаний з необхідністю розв’язання якісно нових задач, що виникають з практичних потреб. Так було при створенні атомної зброї, при освоєнні космосу, моделюванні турбулентності. В якості означення методів Монте-Карло можна навести наступне:
Методи Монте-Карло -- це чисельні методи вирішення математичних задач (систем алгебраїчних, диференційних, інтегральних рівнянь) і пряме статистичне моделювання (фізичних, хімічних, біологічних, економічних, соціальних процесів) за допомогою отримання і перетворення випадкових чисел. Перша робота із застосуванням методу Монте-Карло була опублікована Холом в 1873 році при організації стохастичного процесу експериментального визначення числа $\pi$ шляхом кидання голки на розкреслений аркуш паперу (сама задача теорії геометричних імовірностей про голку була сформульована Ж. Бюффоном ще в 1733 році, а розв’язок до неї був опублікований ним в 1777 році). Один з прикладів використання методу Монте-Карло --а використання ідеї Дж. Фон Неймана при моделюванні траєкторій нейтронів в лабораторії в Лос Аламосі в 1940-х роках. Метод названо на честь столиці князівства Монако, відомої своїми численними  казино, основу яких складає рулетка — досконалий генератор випадкових чисел. Перша робота, де метод Монте-Карло викладався систематично, була опублікована в 1949 році Метрополісом і Уламом, де цей метод застосовувався для для розв’язання лінійних інтегральних рівнянь, що були пов’язані із задачею про проходження нейтронів через речовину \cite{belocerkovskyi}.

\subsection{Методи інтегральних рівнянь при розрахунку електризації}

\section{Математична модель}
Для моделювання руху і зіткнень частинок та космічного апарату було введено низку структур даних, що представляють як геометричні абстракції (точка, пряма, площина, вектор, сфера), так і реальні об’єкти (елементарна частинка, полігональний тривимірний об’єкт). Для роботи з цими структурами було, окрім методів самих структур, розроблено модуль з окремими функціями -- їх опис, а також опис структур даних подано нижче.

\subsection{Опис структур даних}
Для представлення чисел з плаваючою крапкою введено тип real, який є альтернативним іменем для типу float (одинарна точність), а при недостачі точності чи інших потребах може бути легко замінений на double (подвійна точність).

\subsubsection{Point}
Тип потрібен для представлення точки в тривимірному просторі -- в кожному об’єкті зберігаються три координати типу float. Також наявні методи для порівняння з іншими об’єктами цього ж типу і методи для додавання або віднімання вектора (виконується зсув точки на заданий вектор).

\subsubsection{Vector}
Тип введено для представлення вектора в тривимірному просторі. Клас Vector успадковано від класу Point, оскільки вектор теж однозначно задається трьома координатами -- при необхідності може перетворений до батьківського класу. Серед методів можна назвати порівняння з об’єктами цього ж типу, множення вектора на константу і на вектор (в наявності як скалярний добуток, так і векторний, знаходження суми і різниці векторів, довжини вектора, косинуса кута між векторами, а також нормалізація вектора і приведення його до заданої довжини.

\subsubsection{Locus}
Базовий шаблонний клас для всіх підкласів, які представляють собою геометричний об'єкт, що задається набором точок. В якості параметра класу виступає кількість точок. Об’єкт класу містить лише масив заданої параметром довжини і має метод для виводу цього масиву в зручній для сприйняття формі. В перспективі до класу можуть бути додані методи, що працюють відразу з усіма точками, незалежно від їх кількості (наприклад, афінні перетворення).

\subsubsection{Line}
Клас успадковано від Locus з параметром 2, тобто містить в собі дві точки, а також для зручності направляючий вектор, котрий обчислюється при конструюванні об’єкта. Серед методів можна відмітити метод отримання точки на прямій за коефіцієнтом, яким ця точка визначається в параметричних рівняннях прямої.

\subsubsection{ThreePoints}
Базовий клас для всіх класів, що представляють об’єкти, які можуть бути задані трьома точками (трикутник, площина, орієнтована площина). Клас успадковано від Locus з параметром 3, тобто містить в собі три точки. Окрім оператору присвоювання має ще метод для визначення нормалі (за допомогою векторного добутку векторів, утворених двома різними парами точок), а також методи для знаходження центру мас і площі трикутника, утвореного цими трьома точками (при умові, що три точки не лежать на одній прямій).

\subsubsection{Plane}
Клас для представлення площини, успадкований від класу ThreePoints. Додано метод для визначення чи належить точка даній площині.

\subsubsection{OrientedPlane}
Клас для представлення орієнтованої площини, успадкований від класу Plane. Містить вектор нормалі, який тепер повертається перевантаженим метод отримання нормалі, який було визначено в класі ThreePoints. Нормаль обчислюється при конструюванні об’єкта як векторний добуток двох векторів: один з початком в першій точці, кінцем в другій, інший з початком в першій точці, кінцем в третій  -- тобто перший, другий вектори і нормаль утворюють праву трійку векторів; іншими словами з кінця нормалі, початок якої лежить в площині, видно три точки, якими задається площина, в порядку руху годинникової стрілки. Конструктор класу, що описується, приймає також логічний параметр, який визначає, чи буде видно точки в напрямі руху годинникової стрілки (за замовчуванням цей параметр істинний).

\subsubsection{Particle}
Клас, що описує елементарну частинку. Є успадкованим від класу Point. Додано поля для збереження вектора руху, а також поле типу real, що визначає час існування частинки в секундах (якщо значення від’ємне, час вважається не заданим).

\subsubsection{Sphere}
Клас для представлення сфери. Має поля для збереження точки -- центру сфери, а також числа типу real -- радіуса сфери.

\subsubsection{Object3D}
Клас, призначений для збереження координат полігонів тривимірних тіл. При конструюванні кожного об’єкта обчислюються і пишуться у дві відповідні точки (об’єкти Point) максимальні та мінімальні координати тіла за всіма осями. Клас успадковано від класу Sphere -- центр шукається як середина відрізка, що сполучає дві вищеописані точки, радіус -- як половина довжини цього відрізка; таким чином виходить, що ці параметри задають сферу, описану навколо тіла. Це наслідування є корисним при перевірці, чи перетинає траєкторія частинки тіло -- спочатку перевіряється, чи перетинає пряма траєкторія сферу, якщо так -- виконується перевірка для кожного полігону циклічно. Як параметр конструктора об’єкта приймається також вектор, що задає напрям руху тіла (за замовчуванням співпадає з віссю OX). Також в класі визначено метод для пошуку повної поверхні тіла -- вона обчислюється як сума площ всіх полігонів, з яких це тіло складається.

\subsubsection{GenerativeSphere}
Клас, призначений для генерації елементарних частинок. Є нащадком класу Sphere. По суті являє собою сферу, центр якої співпадає з центром тіла, а радіус має перевищувати радіус тіла. Полями класу є генератори випадкових чисел, за допомогою яких для кожної частинки генерується випадковим чином швидкість (один генератор для кожного типу частинок -- іонів та електронів), а також посилання на об’єкт, що представляє собою власне тіло (типу Object3D). В класі наявні методи для генерації частинок наступним чином:
\begin{enumerate}
 \item частинки, що в початковий момент часу свого існування лежать всередині сфери і траєкторія яких не обов’язково перетинає тіло;
 \item частинки, що в початковий момент часу свого існування лежать на сфері і траєкторія яких перетинає тіло;
\end{enumerate}

\subsection{Опис геометричних функцій}
\subsubsection{Перевірка, чи знаходиться точка всередині трикутника}
Перевірка виконується для точки, що знаходиться в площині трикутника.

\textbf{Спосіб 1}

 Виконується перевірка, чи не співпадає точка з однією з вершин трикутника. Після цього шукаються кути між всіма парами векторів, початок яких знаходиться в даній точці, а кінець співпадає з вершиною трикутника. Очевидно, що якщо всі кути будуть тупими, тобто всі косинуси від’ємними, то точка лежатиме всередині трикутника. Також очевидно, що якщо три або два кути виявляться гострими, то точка лежить за межами трикутника. Інакше за допомогою стандартної функції знаходження арккосинуса шукаємо суму всіх кутів. Неважко побачити, що для точок, які не належать трикутнику, ця сума буде менше за $2\pi$.

\textbf{Спосіб 2}

 Для кожної пари вершин трикутника виконаємо перевірку: проведемо через них пряму і визначимо, чи лежить точка і третя вершина в одній і тій самій півплощині відносно цієї прямої. Для цього знайдемо проекцію третьої вершини на пряму і косинус кута, вершиною якого є дана проекція, а сторони проходять через задану точку і третю вершину трикутника. Якщо синус від’ємний, то кут є тупим, тобто точка і третя вершина трикутника лежить в різних півплощинах -- перевірка не виконалась. Якщо для кожної пари вершин ця перевірка виконається, то точка лежатиме всередині трикутника. Інакше -- ні.

\subsubsection{Пошук точки перетину прямої і площини} \label{sec:intersections}
Як відомо, пряма в просторі (тут і далі мається на увазі тривимірний Евклідів простір) може бути задана трьома параметричними рівняннями
\begin{equation} \label{eq:line}
  \begin{cases}
    x = A_x - k(B_x - A_x) \\
    y = A_y - k(B_y - A_y) \\
    z = A_z - k(B_z - A_z)
  \end{cases},
\end{equation}

де A, B -- точки, через які проведено пряму. В цьому випадку кожна її точка задається єдиним значенням коефіцієнта. В обох описаних нижче способах ми знаходимо коефіцієнт точки перетину прямої і площини.

\textbf{Спосіб 1}

В канонічне рівняння площини, проведеної через 3 точки A, B і C
\begin{equation} \label{eq:plane}
  \begin{vmatrix}
    x - A_x & y - A_y & z - A_z \\
    B_x - A_x & B_y - A_y & B_z - A_z \\
    C_x - A_x & C_y - A_y & C_z - A_z \\
  \end{vmatrix} = 0
\end{equation}
підставимо координати з рівнянь \ref{eq:line}. Отримаємо лінійне рівняння відносно коефіцієнта k. Розв’язавши його, отримаємо коефіцієнт шуканої точки перетину.

\textbf{Спосіб 2}

У відоме векторне рівняння площини
\begin{equation} \label{eq:plane-vector}
  \bar n \cdot \overline{AX} = 0
\end{equation}
(де $\bar n$ -- нормаль площини, P -- задана точка на площині, а X -- довільна точка площини) підставляємо замість X точку з координатами, взятими з параметричного рівняння прямої \ref{eq:line}. Отримаємо лінійне рівняння відносно коефіцієнта k. Розв’язавши його, отримаємо коефіцієнт шуканої точки перетину.

\subsubsection{Пошук проекцій} \label{sec:projections}
\textbf{Пошук проекції точки на пряму}

Щоб точка на прямій була проекцією заданої точки, необхідно виконання двох умов:
\begin{enumerate}
  \item Її координати мають задовольняти рівняння прямої \ref{eq:line};
  \item Вектор з точки до її проекції має бути перпендикулярним направляючому вектору прямої: $\overline{PP'} \cdot \bar n = 0$.
\end{enumerate}
Підставляючи замість координат проекції P' координати з рівнянь \ref{eq:line}, отримуємо коефіцієнт точки перетину.

\textbf{Пошук проекції точки на площину}
Шукана точка -- точка перетину прямої (направляючий вектор якої дорівнює нормалі площини; пряма проходить через задану точку) та заданої площини. Пошук перетину прямої і площини описаний в \ref{sec:intersections}.

\subsubsection{Пошук відстаней} \label{sec:distance}
\textbf{Відстань між двома точками}

Знаходиться за відомою формулою
\begin{equation} \label{eq:distance}
  \rho(x,y) = \sqrt{\sum \limits_{i=1}^3 (x_i - y_i)^2}
\end{equation}

\textbf{Відстань між точкою та площиною}

Знаходимо проекцію точки на площину і підставляємо в формулу \ref{eq:distance}.

\subsubsection{Перевірка, чи перетинає пряма сферу}
Неважко побачити, що у випадку, коли пряма перетинає сферу, проекція центру сфери на пряму не буде лежати зовні сфери. Тобто, необхідно лише знайти відстань (див. \ref{sec:distance}) між центром сфери і його проекцією (див. \ref{sec:projections}) на задану пряму та переконатись, що ця відстань не перевищує радіус сфери.

\subsubsection{Поворот точки навколо прямої}
Знайдемо проекцію точки на пряму -- точку P'. Введемо двовимірну систему координат: одиничний вектор (орт) осі абсцис $\bar i$ співпадатиме з вектором $\overline{P'P}$, а для осі ординат $\bar j$ буде одночасно перпендикулярним до осі абсцис та направляючого вектора прямої (знайдемо його як векторний добуток направляючого вектора і першої осі); також вісь ординат нормалізуємо і домножимо на довжину осі абсцис, щоб система була декартовою. Очевидно, що шукана точка знаходиться саме в отриманій площині. Як відомо, на площині координати повороту точки навколо початку координат задаються наступною формулою:
\[
  \begin{cases}
    x' = x \cdot cos \alpha - y \cdot sin \alpha \\
    y' = x \cdot sin \alpha + y \cdot cos \alpha
  \end{cases}
\]
 Також очевидно, що початкова точка в новій системі матиме координати (0;1), отже формула повороту для неї перепишеться як
\[
  \begin{cases}
    x' = - y \cdot sin \alpha \\
    y' = y \cdot cos \alpha
  \end{cases}
\]
Тепер залишилось перейти від двовимірних координат назад до трьохмірних -- для цього необхідно до координат точки P' додати зміщення $x' \bar i + y' \bar j$.

\section{Математичний опис алгоритму}
\subsection{Опис розв’язання}
В даній роботі проводиться пряме моделювання методом Монте-Карло. При такому підході виконується моделювання окремих частин фізичної системи, для прискорення розрахунків допускається застосування деяких фізичних наближень. В нашому випадку елементи фізичної системи — це штучний супутник і велика кількість елементарних частинок, для яких реалізується стохастичний процес їх руху і зіткнення із літальним апаратом.

Для моделювання електризації були взяті характеристики навколишнього середовища геостаціонарної орбіти \cite{novikov}:

-- модель плазми проста Максвелівська:
\renewcommand{\labelitemi}{$\circ$}
\begin{itemize}
 \item Стан системи рівноважний, тому розподіл частинок плазми за абсолютним значенням швидкості відбувається за розподілом Максвела, функція розподілу якого має вигляд
\[
  F(v) = \left(\frac{m}{2 \pi k T}\right)^{3 \over 2} e^\frac{-mv^2}{2 k T} 4 \pi v^2
\]
 де $v$ -- швидкість частинки, $m$ -- маса частинки, $k = 1.38 \cdot 10^-23$ Дж/к -- постійна Больца, $T$ -- температура в градусах Кельвіна.
 \item зовнішні поля відсутні. Доведення: на частинки плазми діє магнітне поле Землі, яке змушує їх рухатись по кільцевим траєкторіям з т.зв. Ларморівським радіусом. Але цим полем можна знехтувати, так як Ларморівський радіус для електронів і іонів (протонів) високотемпературної плазми значно перевищує характерний розмір ШСЗ:
\begin{eqnarray}
 R_{le} = \frac{v_e \perp m_e}{eB} = \frac{3.37 \sqrt E_e}{B} = \frac{3.37 \sqrt {0.4 \cdot 10^3}}{3.1 \cdot 10^-5} = 2.1 \cdot 10^6 m \\
 R_{le} = \frac{v_i \perp m_i}{eB} = \frac{145 \sqrt {A E_i}}{B} = \frac{145 \sqrt {1 \cdot 0.2 \cdot 10^3}}{3.1 \cdot 10^-5} = 6.6 \cdot 10^7 m
\end{eqnarray}
де А -- заряд іона.
 \item рух частинок системи відбувається за напрямками швидкості і є рівномірним.
\end{itemize}
-- частинки рухаються без зіткнень між собою, так як довжина вільного пробігу на геостаціонарній орбіті більша за характерні розміри ШСЗ: на висоті 300 км дорівнює декільком кілометрам.

Отже, рух частинок розглядається як рівномірний прямолінійний, а зіткнення частинок між собою не враховуються, оскільки не мають значення для розв’язання задачі і не впливають на результати.

Елементарні частинки генеруються на сфері, що описується навколо тіла (літального апарату); центр сфери співпадає з центром тіла. Їх координати і напрям задаються випадковим чином, а швидкість -- за допомогою нормального розподілу Максвела. При цьому система має такі параметри: $n$ -- кількість частинок, присутніх в системі в кожну одиницю часу, $R$ -- радіус сфери, на якій будуть генеруватись частинки (між $n$ і $R$ існує залежність, оскільки кількість частинок визначається об’ємом сфери, всередині якої вони знаходяться) і $\Delta t$ -- часовий крок.

\subsection{Опис формату файлу даних}
Модель тіла зчитується з файлу, в якому задаються координати вершин трикутних полігонів.

\section*{Висновок}
\textbf{по пунктам}

\section*{Список використаної літератури}

\section*{Додаток}

\begin{thebibliography}{9}

\bibitem{report1}
  А. М. Капулкин, В. Г. Труш, Д. В. Красношапка:
  \emph{Исследование плазменных нейтрализаторов для снятия электростатических зарядов с поверхности высокоорбитальных космических аппаратов}.
  ДНУ, 1994.

\bibitem{sobol}
  И. М. Соболь:
  \emph{Метод Монте-Карло}.
  «Наука», Москва, 1968.

\bibitem{belocerkovskyi}
  О.М. Белоцерковский, Ю.И. Хлопков:
  \emph{Методы Монте-Карло в прикладной математике и вычислительной аэродинамике}.

\bibitem{sharp}
  Sharp R.D., Shelley E.G., Johonson K.G., Paschmann G.
  \emph{Preliminary Results of a Low Energy Particle Survey at Synchronous Altitude}
  JGR, 1970, Volume 75, P. 6092

\bibitem{deforest}
  DeForest S.E.
  \emph{Spacecraft Charging at Synchronous Altitudes}
  JGR, 1972, Volume 77, P. 651-659

\bibitem{novikov}
  Новиков Л.С.
  \emph{Взаимодействие космических аппаратов с окружающей плазмой}
  Учебное пособие. -- М.: Университетская книга, 2006. -- 120 с.

\end{thebibliography}
      
\end{document}
