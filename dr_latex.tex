\documentclass[a4paper,12pt]{article}
\usepackage [T2A] { fontenc }
\usepackage[utf8]{inputenc}
\usepackage [english] {babel}
\usepackage{indentfirst}

\begin{document}

\section{Розробка програмного забезпечення для обчислення електризації літальних апаратів}
Для штучних супутників (ШС) Землі відомою є проблема електризації. Суть проблеми полягає в нерівномірній електризації поверхні супутника (наприклад, частини поверхні ШС, що знаходяться в сонячній тіні, зряджаються до високого від’ємного потенціала відносно оточуючого космісного простору) \cite{report1}. Через неоднакову освітленість діелектричних ділянок ШС виникає різниця потенціалів, яка веде до електричних пробоїв - вони руйнують поверхню супутника і ведуть до сбоїв в роботі радіоелектронних приладів.

Дана робота присвячена розробці програмного забезпечення для моделювання електризації тіл в космісному просторі, а також обчислення значень електричних потенціалів, що будуть накопичуватися на цих тілах або їх частинах.

Знаючи реальні дані (швидкість електронів, густину їх розподілу, швидкість тіла), проводиться серія випробувань (результати щоразу будуть різними за рахунок використання датчика псевдовипадкових чисел) і отримуються значення зміни потенціалу за проміжок часу.

Моделювання проводится за методом Монте-Карло \cite{report2}. Цей метод заснований на багатократних реалізація стохастичного процесу і застосовується в тих випадках, коли побудова точної математичної моделі неможлива або значно ускладнена. Прояв методів статистичного моделювання в різних областях прикладної математики, як правило, пов’язаний з необхідністю розв’язання якісно нових задач, що виникають з практичних потреб. Так було при створенні атомної зброї, при освоєнні космосу, моделюванні турбулентності. В якості означення методів Монте-Карло можна навести наступне:
Методи Монте-Карло — це чисельні методи вирішення математичних задач (систем алгебраїчних, диференційних, інтегральних рівнянь) і пряме статистичне моделювання (фізичних, хімічних, біологічних, економічних, соціальних процесів) за допомогою отримання і перетворення випадкових чисел. Перша робота із застосуванням методу Монте-Карло була опублікована Холом в 1873 році при організації стохастичного процесу експериментального визначення числа $\pi$ шляхом кидання голки на розкреслений аркуш паперу (сама задача теорії геометричних імовірностей про голку була сформульована Ж. Бюффоном ще в 1733 році, а розв’язок до неї був опублікований ним в 1777 році). Один з прикладів використання методу Монте-Карло — використання ідеї Дж. Фон Неймана при моделюванні траєкторій нейтронів в лабораторії в Лос Аламосі в 1940-х роках. Метод названо на честь столиці князівства Монако, відомої своїми численними  казино, основу яких складає рулетка — досконалий генератор випадкових чисел. Перша робота, де метод Монте-Карло викладався систематично, була опублікована в 1949 році Метрополісом і Уламом, де цей метод застосовувався для для розв’язання лінійних інтегральних рівнянь, що були пов’язані із задачею про проходження нейтронів через речовину \cite{report3}.

В даній роботі проводиться пряме моделювання методом Монте-Карло. При такому підході виконується моделювання окремих частин фізичної системи, для прискорення розрахунків допускається застосування деяких фізичних наближень. В нашому випадку елементи фізичної системи — це штучний супутник і велика кількість елементарних частинок, для яких реалізується стохастичний процес їх руху і зіткнення із літальним апаратом. При цьому рух частинок розглядається як рівномірний прямолінійний, а зіткнення частинок між собою не моделюються, оскільки не мають значення для розв’язання задачі і не впливають на результати.

Елементарні частинки генеруються на сфері, що описується навколо тіла (літального апарату); центр сфери співпадає з центром тіла. Їх координати і напрям задаються випадковим чином, а швидкість - за допомогою нормального розподілу (для цього потрібно знати середню, максимальну і мінімальну швидкість - за правилом трьох сігм знаходяться параметри нормального розподілу). При цьому система має такі параметри: $n$ - кількість частинок, присутніх в системі в кожну одиницю часу, $R$ - радіус сфери, на якій будуть генеруватись частинки (між $n$ і $R$ існує залежність, оскільки кількість частинок визначається об’ємом сфери, всередині якої вони знаходяться) і $\Delta t$ - часовий крок.

Модель тіла зчитується з файлу, в якому задаються координати вершин трикутних полігонів. 

Висновок
Про отримані результати <в процесі>

\begin{thebibliography}{9}

\bibitem{report1}
  А. М. Капулкин, В. Г. Труш, Д. В. Красношапка:
  \emph{Исследование плазменных нейтрализаторов для снятия электростатических зарядов с поверхности высокоорбитальных космических аппаратов}.
  ДНУ, 1994.

\bibitem{report2}
  И. М. Соболь:
  \emph{Метод Монте-Карло}.
  «Наука», Москва, 1968.

\bibitem{report3}
  О.М. Белоцерковский, Ю.И. Хлопков:
  \emph{Методы Монте-Карло в прикладной математике и вычислительной аэродинамике}.

\end{thebibliography}
      
\end{document}
