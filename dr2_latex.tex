\documentclass[a4paper,12pt]{article}
\usepackage [T2A]{fontenc}
\usepackage[utf8]{inputenc}
\usepackage [english,ukrainian] {babel}
\usepackage{indentfirst}
\usepackage{amsmath}
\usepackage{setspace}
\usepackage{enumerate}
\usepackage{url}
\usepackage{lastpage}
\usepackage{graphicx}

\usepackage{listings}

\usepackage{geometry}
\geometry{left=1cm,top=1cm,right=1cm,bottom=1cm,nohead,nofoot}


\begin{document}

\subsection{Обчислення напруженості поля}

Нехай є деяке електростатичне поле і невідома функція $u(x,y,z)$, що задає величину електростатичного потенціалу в кожній точці області простору.

Напруженість і потенціал електростатичного поля пов’язані співвідношенням
\[
E = -\nabla u.
\]
або
\begin{equation} \label{eq:e_to_u}
E_x = -\frac{\partial u}{\partial x}, E_y = -\frac{\partial u}{\partial y}, E_z = -\frac{\partial u}{\partial z}.
\end{equation}

За теоремою Гауса для напруженості електричного поля \cite{magnet}:
\[
\nabla \cdot E = 0
\]
або
\begin{equation} \label{eq:gauss_eq}
\frac{\partial E_x}{\partial x} + \frac{\partial E_y}{\partial y} + \frac{\partial E_z}{\partial z} = 0
\end{equation}

Піставляючи \ref{eq:e_to_u} в \ref{eq:gauss_eq}, отримуємо рівняння, що описує невідому функцію $u(x,y,z)$:
\[
\frac{\partial ^2 u}{\partial x^2} + \frac{\partial ^2 u}{\partial y^2} + \frac{\partial ^2 u}{\partial z^2} = 0,
\] або
\[
\Delta u = 0.
\]

При цьому сумарний потенціал на поверхні тіла в початковий момент буде дорівнювати деякій константі

\[
\left.u\right|_S = 1.
\]

(з огляду на те, що поверхня тіла є добре провідною, можна вважати, що потенціал на поверхні в кожний момент постійний).

Оскільки шукана функція $u$ задана в обмеженій області і відомі її значення на границі цієї області, задача, що розглядається, є крайовою задачею Діріхле.

Введемо функцію \[F_{MP} = \frac{1}{4\pi r_{MP}},\] де \[r_{MP} = \sqrt{(x_M - x_P)^2 + (y_M - y_P)^2 + (z_M - z_P)^2}.\]
Ця функція називається фундаментальним розв’язком тривимірного рівняння Лапласа. Тоді за другою формулою Гріна \cite{ilina}, для довільної точки М області (область поза літальним апаратом, яка обмежується поверхнею літального апарату) буде справедливо:

\[
u_M = \int\limits_S \frac{\partial u}{\partial n}(P) F_{MP} \, d S_P - \int\limits_S u_P \frac{\partial F_{MP}}{\partial n} \, \mathrm{d}S_P
\]

Оскільки поверхня нашого тіла апроксимується трикутними полігонами, можна інтеграли по площі поверхні тіла переписати як суму інтегралів по площі кожного з полігонів (яка є постійною на кожному полігоні):

\begin{equation} \label{eq:triangulated_equation}
u_M = \sum_{j=1}^{N} \frac{\partial u}{\partial n}(j) \int\limits_{S_j} F_{MP} \, dS_j - \sum_{j=1}^{N} u_j \int\limits_{S_j} \frac{\partial F_{MP}}{\partial n} \, dS_j
\end{equation}

($u$ та її похідна можуть бути винесені з-під знаку інтегралу, оскільки на кожному полігоні вони приймають постійне значення).

В останній формулі замість M послідовно підставивши N точок (N -- кількість полігонів, які апроксимують поверхню тіла), кожна з яких є центром одного з полігонів, отримаємо N рівнянь:

\[
u_i = \sum_{j=1}^{N} \frac{\partial u}{\partial n}(j) \int\limits_{S_j} F_{iP} \, dS_j - \sum_{j=1}^{N} u_j \int\limits_{S_j} \frac{\partial F_{iP}}{\partial n} \, dS_j, i = 1..N
\]

Позначимо поверхневі інтеграли по полігонам, що є коефіцієнтами, наступним чином:
\[A_{ij} = \int\limits_{S_j} F_{iP} \, dS_j\]
\[B_{ij} = \int\limits_{S_j} \frac{\partial F_{iP}}{\partial n} \, dS_j.\]

Ці коефіцієнти є постійними для задачі, що розглядається, тому можуть бути обчислені лише раз для кожного тіла і використовуватись в подальшому без змін. Можна побачити, що при $i = j$ в вищезазначених формулах з’являться невласні інтеграли, які належить обчислити окремо.

Отримаємо систему з N алгребраїчних рівнянь відносно $\frac{\partial u}{\partial n}(j)$
\[
u_i = \sum_{j=1}^{N} \frac{\partial u}{\partial n}(j) A_{ij} - \sum_{j=1}^{N} u_j B_{ij}, i = 1..N
\]
або
\[
\sum_{j=1}^{N} \frac{\partial u}{\partial n}(j) A_{ij} = u_i + \sum_{j=1}^{N} u_j B_{ij}, i = 1..N
\]

Розв’язавши дану систему лінійних алгебраїчних рівнянь, отримаємо значення $\frac{\partial u}{\partial n}(j)$. Таким чином, підставивши ці значення в \ref{eq:triangulated_equation}, маємо можливість знайти значення $u_M$ для будь-якої точки області.

Розв’зання вищеописаної задачі Діріхле відбувається у зовнішньому модулі, що написаний на мові Fortran і підключається до програми як об’єктний файл.

\subsection{Опис розв’зання}
Програма дозволяє проводити моделювання як із врахуванням власного електричного поля КА, так і без нього.

Розглянемо, як обчислюється зміна траєкторії частинок при врахуванні електричного поля.

За допомогою підпрограми на фортрані знаходимо вектор градієнту $G$ в бажаній точці. Як відомо, напруженість електричного поля $E$ має зворотній напрямок:
\[
E = -G.
\]

Щоб знайти величину напруженості в точці, застосуємо закон Кулона для знаходження напруженості на відстані $r$ від сфери, що має заряд $q_S$:
\[
|E| = \frac{1}{4 \cdot \pi \cdot \varepsilon_0} \cdot \frac{q_S}{R^2},
\]
де $R$ -- відстань від точки до центру сфери.

Отже, знайшовши вектор напруженості, змінюємо його довжину на цю величину.

Електричне поле діє на частинку із силою
\[
F = q \cdot E,
\]
де $q$ -- заряд частинки.
Звідси, а також із другого закону Ньютона
\[
F = m \cdot a,
\]
знаходимо прискорення частинки:
\[
a = \frac{E \cdot q}{m}.
\]
Знаючи прискорення, знаходимо відстань, що пройде частинка, та її нову швидкість:
\[
S = V \cdot t + G \cdot \frac{q}{m} \cdot \frac{t^2}{2}
\]
\[
V = V_{0} + a \cdot t,
\]
де $t$ -- крок часу, з яким проводиться моделювання.

Оскільки знайдена відстань є векторною величиною, можемо знайти координати нового положення частинки, виконавши зсув попереднього її положення на цей вектор:
\[
P = P_0 + S.
\]

\subsection{Режими роботи програми}
Програма має три режими роботи (бажаний режим задається аргументом командного рядка):
\begin{enumerate}
\item Моделювання без урахування електричного поля космічного апарату -- після створення об’єкту частинки (Particle) за допомогою функції getIndexOfPolygonThatParicleIntersects визначається, чи перетинає траєкторія частинки космічний апарат, а якщо перетинає -- який саме його полігон. Далі знаходиться шлях, що має пройти частинка до зіткнення з апаратом або до того, як вона покине межі генеруючої сфери. На основі цього шляху і швидкості частинки рахується час, який вона повинна існувати. Далі на кожному кроці цей час зменшується на час кроку, а положення частини змінюється відповідно до вектору швидкості. Коли час існування частинки стає недодатнім, вона видаляється з масиву частинок. Заряд апарату збільшується на відповідну величину (заряд частинки помножений на відношення реальної щільності частинок до їх щільності у моделі; це відношення зберігається у змінній Globals::realToModelNumber), якщо частинка перетинає апарат.

\item Моделювання з урахуванням електричного поля космічного апарату (оптимізоване для найшвидшого обчислення) -- на початку роботи програми викликається функція laplace\_, що приймає аргументом масив полігонів космічного апарату і розв’язує крайову задачу, що описана вище, ініціалізуючи внутрішні змінні, необхідні для подальшого розв’язання. В ході моделювання всі частинки помічаються відповідним типом: частинка має невизначену поведінку, частинка зіткнеться з космічним апаратом або частинка не зіткнеться з апаратом (при створенні всі частинки помічаються як такі, поведінка яких не визначена). Для кожної частинки виконуються такі операції:

\begin{itemize}
\item Знаходится відстань від частинки до поверхні космічного апарату. Якщо вона нульова (тобто частинка вже зіткнулась з апаратом) або менша за відстань, яку частинка проходить за один крок часу, то частинка помічається як така, що зіткнеться з апаратом (якщо її траєкторія перетинає апарат) або як така, що не зіткнеться (якщо не перетинає). При цьому заряд частинок і електричне поле апарату не беруться до уваги, бо вважається, що він уже не встигне істотно змінити траєкторію частинки.

\item Якщо відстань до апарату велика (більше одного кроку), то для точки, в якій знаходиться частинка обчислюється градієнт і електричний потенціал за допомогою функції resultf\_ (що використовує обчислені на початку програми функцією laplace\_ коефіцієнти). Далі за описаним алгоритмом положення і швидкість частинки, на яку діє електричне поле, змінюється. Далі за допомогою функції getIndexOfPolygonThatParicleIntersects визначається, чи перетинає траєкторія частинки апарат. Якщо перетинає, і заряд частинки за знаком протилежний заряду апарату, то частинка помічається як така, що зіткнеться з апаратом. Час, який частинка буде існувати, обчислюється як час, через який вона зіткнеться з апаратом. Якщо ж частинка не перетинає апарат і має заряд того ж знаку, що й апарат, то вона помічається як така, що не перетне апарат.

Слід також зазначити, що частинки видаляються з масиву (знищуються) тоді, коли час їх існування (час, відведений їм на існування) стає недодатнім, або коли частинка полишає межі генеруючої сфери. Тому коли ми знаємо, що частинка не зіткнеться з апаратом, можемо виставити їй досить великий час існування, і її буде видалено, як тільки вона вилетить за межі сфери.

З метою прискорення обчислень, траєкторія частинок, помічених як такі, що перетинають або не перетинаються сферу (на відміну від частинок, поведінка яких ще не визначена, і на траєкторію яких ще може вплинути електричне поле апарату), змінюється так само, як в першому режимі роботи, тобто змінюється лише час існування (час, що залишився) і положення частинки.
\end{itemize}

\item Моделювання з урахуванням електричного поля космічного апарату (найбільше підходить для візуалізації процесу) -- цей режим подібний до другого режиму, але частинки, що знаходяться більше, ніж на один крок від апарату, ніколи не помічаються як такі, що зіткнуться або не зіткнуться з апаратом. Це зроблено для того, щоб наочно продемонструвати, як траєкторія частинок змінюється під дією електричного поля апарату (у другому режимі цього не видно, бо коли стає очевидно, що частинка зіткнеться або не зіткнеться з апаратом, вона помічається відповідним чином і її траєкторія замінюється прямолінійною).

\end{enumerate}



\begin{thebibliography}{9}

\bibitem{magnet}
    А.М. Макаров, Л.А. Луньова
    \emph{Основи електромагнетизму}
    МДТУ ім. Н.Е. Баумана  
      
\bibitem{ilina}
	Ільїн А. М.
	\emph{Рівняння математичної фізики}
	Челябінськ, Челябінський державний університет, 2005
	


\end{thebibliography}


\end{document}
