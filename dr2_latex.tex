\documentclass[a4paper,12pt]{article}
\usepackage [T2A]{fontenc}
\usepackage[utf8]{inputenc}
\usepackage [english,ukrainian] {babel}
\usepackage{indentfirst}
\usepackage{amsmath}
\usepackage{setspace}
\usepackage{enumerate}
\usepackage{url}
\usepackage{lastpage}
\usepackage{graphicx}

\usepackage{listings}

\usepackage{geometry}
\geometry{left=1cm,top=1cm,right=1cm,bottom=1cm,nohead,nofoot}


\begin{document}

\subsection{Обчислення напруженості поля}

Нехай є деяке електростатичне поле і невідома функція $u(x,y,z)$, що задає величину електростатичного потенціалу в кожній точці області простору.

Напруженість і потенціал електростатичного поля пов’язані співвідношенням
\[
E = -\nabla u.
\]
або
\begin{equation} \label{eq:e_to_u}
E_x = -\frac{\partial u}{\partial x}, E_y = -\frac{\partial u}{\partial y}, E_z = -\frac{\partial u}{\partial z}.
\end{equation}

За теоремою Гауса для напруженості електричного поля \cite{magnet}:
\[
\nabla \cdot E = 0
\]
або
\begin{equation} \label{eq:gauss_eq}
\frac{\partial E_x}{\partial x} + \frac{\partial E_y}{\partial y} + \frac{\partial E_z}{\partial z} = 0
\end{equation}

Піставляючи \ref{eq:e_to_u} в \ref{eq:gauss_eq}, отримуємо рівняння, що описує невідому функцію $u(x,y,z)$:
\[
\frac{\partial ^2 u}{\partial x^2} + \frac{\partial ^2 u}{\partial y^2} + \frac{\partial ^2 u}{\partial z^2} = 0,
\] або
\[
\Delta u = 0.
\]

При цьому сумарний потенціал на поверхні тіла в початковий момент буде дорівнювати деякій константі

\[
\left.u\right|_S = 1.
\]

(з огляду на те, що поверхня тіла є добре провідною, можна вважати, що потенціал на поверхні в кожний момент постійний)


Оскільки шукана функція $u$ задана в обмеженій області і відомі її значення на границі цієї області, задача, що розглядається, є крайовою задачею Діріхле.

Фундаментальний розв’язок тривимірного рівняння Лапласа: $F_{mp} = \frac{1}{4\pi r_{mp}}$, де \[r_{mp} = \sqrt{(x - x_0)^2 + (y - y_0)^2 + (z - z_0)^2}.\]

(посилання на Тіхнова, Самарського - рівняння мат. фізики)

Для довільної точки М області, що розглядається, буде справедливо:

// \textbf{TODO} Розписати, як ми це отримали
\[
u_M = \int\limits_S \frac{\partial u}{\partial n}(P) F_{MP} \, \mathrm{d}S_P - \int\limits_S u_P \frac{\partial F_{MP}}{\partial n} \, \mathrm{d}S_P
\]

Оскільки поверхня нашого тіла апроксимується трикутними полігонами, можна інтеграли по площі поверхні тіла переписати як суму інтегралів по площі кожного з полігонів (постійна на кожному полігоні):

\begin{equation} \label{eq:triangulated_equation}
u_M = \sum_{j=1}^{N} \frac{\partial u}{\partial n}(j) \int\limits_{S_j} F_{MP} \, \mathrm{d}S_j - \sum_{j=1}^{N} u_j \int\limits_{S_j} \frac{\partial F_{MP}}{\partial n} \, \mathrm{d}S_j
\end{equation}

($u$ та її похідна можуть бути винесені з-під інтегралу, оскільки на кожному полігоні вони приймають постійне значення).

В останній формулі замість M послідовно підставивши N точок (N -- кількість полігонів, які апроксимують поверхню тіла), кожна з яких є центром одного з полігонів, отримаємо N рівнянь:

\[
u_i = \sum_{j=1}^{N} \frac{\partial u}{\partial n}(j) \int\limits_{S_j} F_{iP} \, \mathrm{d}S_j - \sum_{j=1}^{N} u_j \int\limits_{S_j} \frac{\partial F_{iP}}{\partial n} \, \mathrm{d}S_j, i = 1..N
\]

Позначимо
\[A_{ij} = \int\limits_{S_j} F_{iP} \, \mathrm{d}S_j\]
\[B_{ij} = \int\limits_{S_j} \frac{\partial F_{iP}}{\partial n} \, \mathrm{d}S_j.\]

Ці коефіцієнти є постійними для задачі, що розглядається, тому можуть бути обчислені лише раз для кожного тіла і використовуватись в подальшому без змін. Можна побачити, що при $i = j$ в вищезазначених формулах з’являться невласні інтеграли, які належить обчислити окремо.

Отримаємо систему з N алгребраїчних рівнянь відносно $\frac{\partial u}{\partial n}(j)$
\[
u_i = \sum_{j=1}^{N} \frac{\partial u}{\partial n}(j) A_{ij} - \sum_{j=1}^{N} u_j B_{ij}, i = 1..N
\]
або
\[
\sum_{j=1}^{N} \frac{\partial u}{\partial n}(j) A_{ij} = u_i + \sum_{j=1}^{N} u_j B_{ij}, i = 1..N
\]

Розв’язавши дану систему лінійних алгебраїчних рівнянь, отримаємо значення $\frac{\partial u}{\partial n}(j)$. Таким чином, підставивши ці значення в \ref{eq:triangulated_equation}, можна знайти значення $u_M$ для будь-якої точки області.

//задача дирихле, $u$, $u_i$ на границе известно

//интеграл $B_{ij}$ - телесный угол, под которым полигон виден из точки М

// интегральное уравнение Фернгольда 1-2 рода (1 рода - некорректно - но для случая гладких ядер. если i=j, то расстояние между ядрами 0)

\subsection{Опис розв’зання}
Програма дозволяє проводити моделювання як із врахуванням власного електричного поля КА, так і без нього.

Розглянемо, як обчислюється зміна траєкторії частинок при врахуванні електричного поля.

За допомогою підпрограми на фортрані знаходимо вектор градієнту $G$ в бажаній точці. Як відомо, напруженість електричного поля $E$ має зворотній напрямок:
\[
E = -G.
\]

Щоб знайти величину напруженості в точці, застосуємо закон Кулона для знаходження напруженості на відстані $r$ від сфери, що має заряд $q_S$:
\[
|E| = \frac{1}{4 \cdot \pi \cdot \varepsilon_0} \cdot \frac{q_S}{R^2},
\]
де $R$ -- відстань від точки до центру сфери.

Отже, знайшовши вектор напруженості, змінюємо його довжину на цю величину.

Електричне поле діє на частинку із силою
\[
F = q \cdot E,
\]
де $q$ -- заряд частинки.
Звідси, а також із другого закону Ньютона
\[
F = m \cdot a,
\]
знаходимо прискорення частинки:
\[
a = \frac{E \cdot q}{m}.
\]
Знаючи прискорення, знаходимо відстань, що пройде частинка, та її нову швидкість:
\[
S = V \cdot t + G \cdot \frac{q}{m} \cdot \frac{t^2}{2}
\]
\[
V = V_{0} + a \cdot t,
\]
де $t$ -- крок часу, з яким проводиться моделювання.

Оскільки знайдена відстань є векторною величиною, можемо знайти координати нового положення частинки, виконавши зсув попереднього її положення на цей вектор:
\[
P = P_0 + S.
\]

\subsection{Режими роботи програми}
Програма має три режими роботи (бажаний режим задається аргументом командного рядка):
\begin{enumerate}
\item 1
\end{enumerate}



\begin{thebibliography}{9}

\bibitem{magnet}
    А.М. Макаров, Л.А. Луньова
    \emph{Основи електромагнетизму}
    МДТУ ім. Н.Е. Баумана  
      


\end{thebibliography}


\end{document}
