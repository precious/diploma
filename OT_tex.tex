\documentclass[a4paper,12pt]{article}
\usepackage [T2A]{fontenc}
\usepackage[utf8]{inputenc}
\usepackage [english,ukrainian] {babel}
\usepackage{indentfirst}
\usepackage{amsmath}
\usepackage{setspace}
\usepackage{enumerate}
\usepackage{url}
\usepackage{lastpage}
\usepackage{graphicx}

\usepackage{listings}

\usepackage{geometry}
\geometry{left=3.5cm,top=2cm,right=1cm,bottom=2cm,nohead,nofoot}

\usepackage{titlesec}
\titleformat{\section}
  {\normalfont\Large\bfseries\centering}{\thesection}{1em}{}
\titleformat{\subsection}
  {\normalfont\large\bfseries\centering}{\thesubsection}{1em}{}
\titleformat{\subsubsection}
  {\normalfont\normalsize\bfseries\centering}{\thesubsubsection}{1em}{}

\usepackage{fancyhdr}
\fancyhf{}
\lhead{} \chead{} \rhead{\normalsize\thepage}
\lfoot{} \cfoot{} \rfoot{}
\renewcommand{\headrulewidth}{0pt}
\renewcommand{\footrulewidth}{0pt}
\pagestyle{fancy}


\begin{document}
\onehalfspacing
\large

\section{Охорона праці та безпека в надзвичайних ситуаціях}

Люди, основним робочим інструментом яких є комп’ютер, проводять за ним майже весь свій робочий час. При цьому навіть незначний вплив шкідливих та небезпечних факторів може призвести до професійних захворювань, оскільки такі фактори діють впродовж тривалого часу. Тому надзвичайно важливим є дотримання вимог нормативних документів щодо обладнання робочих місць користувачів ПК, роботи із застосуванням ПК, які розроблено згідно з  науковими дослідженнями у сфері безпечної організації робіт з експлуатації ПК та з урахуванням положень міжнародних нормативно-правових актів з цих питань.

!!! тут би не завадив ще абзац води

\subsection{Характеристики робочого приміщення}
Робота виконується на підприємстві, що займається розробкою програмного забезпечення. Розглядається одне з трьох робочих приміщень. Площа приміщення складає 11.7 $\text{м}^2$. Висота приміщення складає 3.2 м. Об’єм повітря -- 37.4 $\text{м}^2$. Кількість людей, що працюють у приміщенні -- двоє. Отже, на одну людину припадає 6 $\text{м}^2$ площі.

Кожне робоче місце складають:
\begin{enumerate}
\item стіл
\item монблочний ПК iMac12.1 MC812XX/A
\item клавіатура Logitech K200
\item оптична миша Sven RX-111
\item навушники Plantronics 655 DSP
\item джерело безперебійного живлення APC Back-UPS ES 525VA
\item крісло поворотне з можливістю регулювання висоти крісла на нахилу спинки
\end{enumerate}

Висота стола складає 165 см, кришка стола має розміри 75х150 см.

Максимально можлива висота крісла -- 60 см, мінімально можлива -- 48 см

Всі ПК під’єднані до локальної бездротової мережі Wi-Fi, яка функціонує на основі бездротового маршрутизатора, що знаходиться у сусідньому приміщенні.

У горизонтальному перерізі приміщення має форму прямокутника зі сторонами 6.1 м та 1.92 м. У меншій стіні розташоване вікно висотою 165 см та шириною 170 см, що виходить на вулицю. У більшій стіні розташоване вікно  висотою 155 см та шириною 165 см, що виходить у сусіднє приміщення. Вздовж цієї стіни розташовані столи.

Вікно на вулицю направлене на північний захід.

При недостачі денного світла приміщення освітлюється лампами денного світла -- над кожним робочим місцем розташовано по 4 лампи.

Стіни приміщення пофарбовані білою фарбою, стеля вкрита пластиковими панелями білого кольору розміром 60х60 см. Підлога вкрита лінолеумом сірого кольору.

!!! впилить инфу про кондер  и про мироклимат

Серед шкідливих та небезпечних факторів, що діють у приміщенні, можна назвати наступні:
\begin{enumerate}
\item Недостатній рівень штучного освітлення 
\item Шум
\item Електричний струм
\item Пожежонебезпека
\item Мікроклімат
\end{enumerate}

!!! мб впилить план эвакуации
\subsection{Аналіз відповідності робочого приміщення встановленим нормам}

Площу  та  об'єм  для  одного  робочого місця оператора визначають згідно з вимогами \cite{sanpin798}. Площа має бути не менше 6.0 $\text{м}^2$, об'єм -- не менше 20.0 $\text{м}^3$. Приміщення, що розглядається, не задовольняє цим нормам, оскільки на одну людину припадає 5.85 $\text{м}^2$ площі і 18.7 $\text{м}^3$ об’єму.

Приміщення знаходиться на першому поверсі, а отже його розташування не суперечить \cite{npaop1210}, де забороняється розташування приміщень з робочими місцями  операторів у підвалах і цокольних поверхах.

Згідно \cite{npaop1210}, заземлені конструкції, що знаходяться в приміщеннях, де розміщені робочі місця операторів (батареї опалення,  водопровідні труби, кабелі із заземленим відкритим екраном), мають бути надійно захищені діелектричними щитками або сітками  з  метою  недопущення потрапляння працівника під напругу. Але в приміщенні, що розглядається, батарея знаходиться під металічним щитком, тобто потрапляння працівника під напругу не виключене. Отже, для відповідності приміщення нормам охорони праці металеві щитки батарей необхідно замінити на діелектричні.

Згідно \cite{npaop1210}, приміщення, де розміщені робочі місця операторів, крім приміщень, у яких розміщені робочі місця операторів великих ЕОМ загального призначення (сервер), мають бути оснащені системою автоматичної пожежної сигналізації. У приміщенні, що описується, системи автоматичної пожежної сигналізації відсутні. Їх необхідно встановити для відповідності приміщення нормам охорони праці.

Згідно \cite{npaop1210}, приміщення, де розміщені робочі місця операторів, крім приміщень, у яких розміщені робочі місця операторів великих ЕОМ загального призначення (сервер), мають бути оснащені вогнегасниками, кількість яких визначається згідно з вимогами Типових норм належності вогнегасників. У приміщенні, що описується, вогнегасники відсутні. Їх необхідно встановити для відповідності приміщення нормам охорони праці.



\begin{thebibliography}{9}

\bibitem{npaop1210}
	НПАОП 0.00-1.28-10
	
\bibitem{sanpin798}
	ДСанПіН  3.3.2-007-98


\end{thebibliography}



\end{document}