\documentclass[a4paper,12pt]{article}
\usepackage [T2A]{fontenc}
\usepackage[utf8]{inputenc}
\usepackage [english,ukrainian] {babel}
\usepackage{indentfirst}
\usepackage{amsmath}
\usepackage{setspace}
\usepackage{enumerate}
\usepackage{url}
\usepackage{lastpage}
\usepackage{graphicx}

\usepackage{listings}

\usepackage{geometry}
\geometry{left=3.5cm,top=2cm,right=1cm,bottom=2cm,nohead,nofoot}

\usepackage{titlesec}
\titleformat{\section}
  {\normalfont\Large\bfseries\centering}{\thesection}{1em}{}
\titleformat{\subsection}
  {\normalfont\large\bfseries\centering}{\thesubsection}{1em}{}
\titleformat{\subsubsection}
  {\normalfont\normalsize\bfseries\centering}{\thesubsubsection}{1em}{}

\usepackage{fancyhdr}
\fancyhf{}
\lhead{} \chead{} \rhead{\normalsize\thepage}
\lfoot{} \cfoot{} \rfoot{}
\renewcommand{\headrulewidth}{0pt}
\renewcommand{\footrulewidth}{0pt}
\pagestyle{fancy}


\begin{document}
\onehalfspacing
\large

\section{Охорона праці та безпека в надзвичайних ситуаціях}

Люди, основним робочим інструментом яких є ЕОМ, проводять за нею майже весь свій робочий час. При цьому навіть незначний вплив шкідливих та небезпечних факторів може призвести до професійних захворювань, оскільки такі фактори діють впродовж тривалого часу. Тому надзвичайно важливим є дотримання вимог нормативних документів, які розроблено згідно з  науковими дослідженнями у сфері безпечної організації робіт з експлуатації ЕОМ та з урахуванням положень міжнародних нормативно-правових актів з цих питань, щодо обладнання робочих місць користувачів ЕОМ, роботи із застосуванням ЕОМ.

\subsection{Характеристики робочого приміщення}
Робота виконується на підприємстві, що займається розробкою програмного забезпечення. Розглядається одне з трьох робочих приміщень. Площа приміщення складає 11.7 $\text{м}^2$. Висота приміщення складає 3.2 м. Об’єм повітря -- 37.4 $\text{м}^2$. Кількість людей, що працюють у приміщенні -- двоє. Отже, на одну людину припадає 5.85 $\text{м}^2$ площі.

Кожне робоче місце складають:
\begin{enumerate}
\item Стіл
\item Монблочний ПК iMac12.1 MC812XX/A
\item Клавіатура Logitech K200
\item Оптична миша Sven RX-111
\item Навушники Plantronics 655 DSP
\item Джерело безперебійного живлення APC Back-UPS ES 525VA
\item Крісло поворотне з можливістю регулювання висоти крісла на нахилу спинки
\end{enumerate}

Висота стола складає 165 см, кришка стола має розміри 75х150 см.

Максимально можлива висота крісла -- 60 см, мінімально можлива -- 48 см.

Всі ЕОМ під’єднані до локальної бездротової мережі Wi-Fi, яка функціонує на основі бездротового маршрутизатора, що знаходиться у сусідньому приміщенні.

У горизонтальному перерізі приміщення має форму прямокутника зі сторонами 6.1 м та 1.92 м. У меншій стіні розташоване вікно висотою 165 см та шириною 170 см, що виходить на вулицю. У більшій стіні розташоване вікно  висотою 155 см та шириною 165 см, що виходить у сусіднє приміщення. Вздовж цієї стіни розташовані столи.

Вікно на вулицю направлене на північний захід.

При недостачі денного світла приміщення освітлюється лампами денного світла -- над кожним робочим місцем розташовано по 4 лампи.

Стіни приміщення пофарбовані білою фарбою, стеля вкрита пластиковими панелями білого кольору розміром 60х60 см. Підлога вкрита лінолеумом сірого кольору.

Приміщення також обладнане кондиціонером Neoclima NS12LHB.

\subsection{Шкідливі та небезпечні виробничі фактори}
Серед шкідливих та небезпечних факторів, що діють у приміщенні, можна назвати наступні \cite{gost_fact}:
\begin{enumerate}
\item Пряме та відбите випромінювання, що створює засліплюючу дію
\item Підвищена чи знижена рухомість повітря
\item Розумове перенапруження
\item Перенапруження органів чуття
\item Електричний струм
\item Пожежонебезпека
\item Електромагнітне випромінювання
\end{enumerate}

Наслідком дії несприятливих виробничих факторів може бути професійне захворювання — патологічний стан людини, обумовлений роботою і пов’язаний з надмірним напруженням організму або несприятливою дією шкідливих виробничих факторів.

Серед професійних захворювань людей, що працюють в ЕОМ, можна виділити такі:
\begin{enumerate}
\item Астноптичні скарги -- викликані функціональними змінамми нервово-м’язового апарата і кровопостачання ока внаслідок роботи з ВДТ, а також проблемами з освітленням робочого місця, відблиском екрану, тремтінням та мерехтінням зображення, сухістю повітря.
\item Розлади функцій шлунково-кишкового тракту, серцево-судинної системи, скелетних м’язів, залоз внутрішньої секреції, шкіри, статевої системи -- викликані фізіологічними порушеннями; частіше мають місце у працівників з високою та середньою тривалістю роботи за ЕОМ.
\item Психологічні та поведінкові розлади -- агресивність, нервозність, дратівливість, порушення сну, швидкий розвиток втоми.
\end{enumerate}

\subsection{Аналіз відповідності робочого приміщення встановленим нормам}
\subsubsection{Відповідність вимогам до виробничих приміщень}
Площу  та  об'єм  для  одного  робочого місця оператора визначають згідно з вимогами \cite{sanpin798}. Площа має бути не менше 6.0 $\text{м}^2$, об'єм -- не менше 20.0 $\text{м}^3$. Приміщення, що розглядається, не задовольняє цим нормам, оскільки на одну людину припадає 5.85 $\text{м}^2$ площі і 18.7 $\text{м}^3$ об’єму.

Приміщення знаходиться на першому поверсі, а отже його розташування не суперечить \cite{npaop1210}, де забороняється розташування приміщень з робочими місцями  операторів у підвалах і цокольних поверхах.

Згідно \cite{npaop1210}, заземлені конструкції, що знаходяться в приміщеннях, де розміщені робочі місця операторів (батареї опалення,  водопровідні труби, кабелі із заземленим відкритим екраном), мають бути надійно захищені діелектричними щитками або сітками  з  метою  недопущення потрапляння працівника під напругу. Але в приміщенні, що розглядається, батарея знаходиться під металічним щитком, тобто потрапляння працівника під напругу не виключене. Отже, для відповідності приміщення нормам охорони праці металеві щитки батарей необхідно замінити на діелектричні.

Згідно \cite{npaop1210}, приміщення, де розміщені робочі місця операторів, крім приміщень, у яких розміщені робочі місця операторів великих ЕОМ загального призначення (сервер), мають бути оснащені системою автоматичної пожежної сигналізації. У приміщенні, що описується, системи автоматичної пожежної сигналізації відсутні. Їх необхідно встановити для відповідності приміщення нормам охорони праці.

Згідно \cite{npaop1210}, приміщення, де розміщені робочі місця операторів, крім приміщень, у яких розміщені робочі місця операторів великих ЕОМ загального призначення (сервер), мають бути оснащені вогнегасниками, кількість яких визначається згідно з вимогами Типових норм належності вогнегасників. У приміщенні, що описується, вогнегасники відсутні. Їх необхідно встановити для відповідності приміщення нормам охорони праці.

Згідно \cite{npaop1210}, ЕОМ з ВДТ і ПП повинні підключатися до електромережі тільки за допомогою справних штепсельних з'єднань і електророзеток заводського виготовлення. У штепсельних з'єднаннях та електророзетках, крім контактів фазового та нульового робочого провідників, мають бути спеціальні контакти для підключення нульового захисного провідника. Їхня конструкція має бути такою, щоб приєднання нульового захисного провідника відбувалося раніше, ніж приєднання фазового та нульового робочого провідників. Порядок роз'єднання при відключенні має бути зворотним. У приміщенні, що розглядається, ЕОМ під’єднуються до мережі через джерело безперебійного живлення. І в електророзетках, і в штепсельних з’єднаннях джерел безперебійного живлення присутні спеціальні контакти для підключення нульового захисного провідника. Їхня конструкція відповідає вимогам \cite{npaop1210}.

Згідно \cite{sanpin798}, віконні прорізи приміщень для роботи з ВДТ мають бути обладнані регульованими пристроями (жалюзі, завіски, зовнішні козирки). Але в приміщенні, що розглядається, такі пристрої відсутні, через що пряме сонячне світло робить незручною роботу з ВДТ. Тому для відповідності нормам, необхідно встановити один з вищезазначених пристроїв.

Згідно \cite{sanpin798}, у приміщеннях з ВДТ слід щоденно робити вологе прибирання. У приміщенні ж, що розглядається, вологе прибиранні робиться 1-2 рази на тиждень. Отже, для відповідності нормам, потрібно частіше робити вологе прибирання.

Також для відповідності приміщення нормам, необхідно оснастити його аптечкою першої медичної допомоги \cite{sanpin798}.

Поряд з приміщенням, що розглядається, знаходиться кімната психологічного розвантаження, в якій є місце для занять фізичною культурою \cite{snip}.

\subsubsection{Відповідність вимогам до ЕОМ з ВДТ і ПП}
ЕОМ відповідають вимогам національних стандартів держав-виробників і мають в експлуатаційній документації відповідну позначку, як і вимагається в \cite{npaop1210}.

\subsubsection{Відповідність вимогам до організації робочого місця оператора}
Згідно \cite{npaop1210}, за потреби особливої концентрації уваги під час виконання робіт суміжні робочі місця операторів необхідно відділяти одне від одного перегородками висотою 1,5 - 2 м. У приміщенні, що описується, відсутні перегородки між робочими місцями, хоча у працівників виникає потреба особливої концентрації уваги. Отже, для відповідності приміщення нормативним вимогам, такі перегородки мають бути встановлені.

Як було описано, робочі столи з ЕОМ розташовані таким чином, що природне світло падає зліва \cite{sanpin798}.

Згідно \cite{sanpin798}, при розміщенні робочих столів з ВДТ слід дотримувати відстань між бічними поверхнями ВДТ 1.2 м. Ця вимога не дотримується -- відстань складає 1 м. Тому для відповідності нормам необхідно збільшити відстань.

Також згідно \cite{sanpin798}, висота робочої поверхні робочого столу з ВДТ має регулюватися в межах 680...800 мм, а ширина і глибина - забезпечувати можливість виконання операцій у зоні досяжності моторного поля (рекомендовані розміри: 600...1400 мм, глибина - 800...1000 мм). Але як було сказано вище, висота столів не регулюється, а глибина менша мінімальної рекомендованої. Отже, для відповідності столів нормам рекомендовано їх замінити.

Крісла повністю відповідають нормам, описаним в \cite{sanpin798}.

Робоче місце має бути обладнане підставкою для ніг, характеристики якої подані в \cite{sanpin798}. Але в приміщенні, що розглядається, такі підставки відсутні. Отже, можна рекомендувати обладнати приміщення такими підставками.

Екран ВДТ знаходиться на відстані 60 см від очей користувача, що узгоджується з допустимою відстанню в \cite{sanpin798}.

Клавіатура розташовується на відстані 30 см від краю стола, що також узгоджується з \cite{sanpin798}.

За характером трудової діяльності робітники, що працюють в приміщенні, належать до першої професійної групи згідно з діючим класифікатором професій (ДК-003-95 і Зміна № 1 до ДК-003-95) -- розробники програм (інженери-програмісти) -- виконують роботу переважно з відеотерміналом та документацією при необхідності та інтенсивного обміну інформацією з ЕОМ і високою частиною прийняття рішень. Робота характеризується інтенсивною розумовою творчою працею з підвищеним напруженням зору, концентрацією уваги на фоні нервово-емоційного напруження, вимушеною робочою позою, загальною гіподинамією, періодичним навантаженням на кисті верхніх кінцівок. Робота виконується в режимі діалогу з ЕОМ у вільному темпі з періодичним пошуком помилок в умовах дефіциту часу. Робітникам можна дати рекомендацію дотримуватися режиму праці при роботі з ЕОМ, як описано в \cite{sanpin798}.

\subsubsection{Відповідність вимогам безпеки під час роботи з ЕОМ з ВДТ і ПП}
Згідно \cite{npaop1210}, після закінчення роботи ЕОМ з ВДТ і ПП повинні бути відключені від електричної мережі. Ця вимога не виконується, оскільки для збереження сеансу роботи ЕОМ переводяться в режим зниженого електроспоживання.

\subsection{Розрахунок пристрою заземлення в заданому типі ґрунту}
Розрахувати захисне заземлення в садовому ґрунті. Опір природного заземлювача складає $R_\text{П}=10\ \text{Ом}$, допустимий опір заземлювача $R_\text{Д}=4\ \text{Ом}$, питомий опір садового ґрунту $\rho=50\ \text{Ом}\cdot\text{м}$ \cite{burakova}. Коефіцієнт сезонності $\psi = 1.3$. Глибина залягання електрода $h = 0.6\ \text{м}$.

Контур заземлення виконують зі сталевих прутів. В траншеї глибиною до 0.7 м вертикально забиваються стрижні, а верхні кінці, що виступають із землі, з’єднуються зварюванням сталевою смугою або прутом.

При цьому необхідно дотримуватися наступних умов \cite{dzunzuk}:
\begin{enumerate}
\item Переріз з’єднувальної смуги має бути не менше 48 $\text{мм}^2$, товщина -- не менше 4 мм.
\item Мінімальний діаметр пруту -- 10 мм.
\item Довжина стрижня має бути не менше 1.5..2 м, щоб досягти незамерзаючого шару ґрунту.
\end{enumerate}

Виходячи з цього, в якості вертикальних електродів візьмемо прути діаметром $d = 0.015\ \text{м}$ і довжиною $l = 3\ \text{м}$.

Знаходимо допустимий опір штучного заземлювача:
\[
R_\text{ш} = \frac{R_\text{П} \cdot R_\text{Д}}{R_\text{П} - R_\text{Д}} = 
\frac{10 \cdot 4}{10 - 4} = 6.67\ \text{Ом}.
\]

Знаходимо відстань від поверхні землі до середини вертикального електрода:
\[
t = h + \frac{l_\text{В}}{2} = 0.6 + 1.5 = 2.1\ \text{м}.
\]

Приймаємо відстань між вертикальними електродами $a = 3\ \text{м}$.

Знаходимо опір одиничного вертикального заземлювача за \cite{dzunzuk}:
\[
R_\text{В} = \frac{\rho\cdot\psi}{2\cdot\pi\cdot l_\text{В}} \cdot \left( ln \frac{2\cdot l_\text{В}}{d} + \frac{1}{2}\cdot ln \frac{4 \cdot t + l_\text{В}}{4 \cdot t - l_\text{В}} \right)
\]
\[
R_\text{В} = \frac{50\cdot1.3}{2\cdot3.14\cdot 3} \cdot \left( ln \frac{2\cdot 3}{0.015} + \frac{1}{2}\cdot ln \frac{4 \cdot 2.1 + 3}{4 \cdot 2.1 - 3} \right) = 22\ \text{Ом}.
\]

Знаходимо орієнтовне число вертикальних заземлювачів:
\[
n_\text{орієнт} = \frac{R_\text{В}}{R_\text{Ш}} = \frac{22}{6.67} = 3.3.
\]

Знаходимо за \cite{dzunzuk} орієнтовний коефіцієнт використання вертикальних електродів:
\[
\eta_\text{В}^\text{орієнт} = \frac{2.02}{n_\text{орієнт}} + 0.346 = 0.96.
\]

Знаходимо число вертикальних заземлювачів:
\[
n = \frac{R_\text{В}}{R_\text{ш}\cdot \eta_\text{В}^\text{орієнт}} = \frac{22}{6.67\cdot 0.96} = 3.4.
\]

Округляємо число електродів до 4 і знаходимо коефіцієнт використання вертикальних електродів:
\[
\eta_\text{В} = \frac{2.02}{n} + 0.346 = 0.85.
\]

Знаходимо довжину горизонтального електрода. При розташуванні електродів в ряд вона дорівнюватиме:
\[
l_\text{Г} = a \cdot (n - 1) = 3 \cdot (4 - 1) = 9\ \text{м}.
\]

Приймаємо товщину горизонтального електрода $b = 0.005\ \text{м}$.

Знаходимо опір горизонтального електрода:
\[
R_\text{Г} = \frac{\rho \cdot \psi}{2 \cdot \pi \cdot l_\text{Г}} \cdot ln \frac{2 \cdot l_\text{Г}^2}{b \cdot h}
\]
\[
R_\text{Г} = \frac{50 \cdot 1.3}{2 \cdot 3.14 \cdot 9} \cdot ln \frac{2 \cdot 9^2}{0.005 \cdot 0.6} = 12.5\ \text{Ом}.
\]

Знаходимо за \cite{dzunzuk} коефіцієнт використання горизонтального електрода:
\[
\eta_\text{Г} = \frac{1.5}{n} + 0.176 = 0.4.
\]

Знаходимо опір штучного заземлювача:
\[
R_\text{Ш} = \frac{R_\text{В} \cdot R_\text{Г}}{R_\text{В} \cdot \eta_\text{Г} + n \cdot R_\text{Г} \cdot \eta_\text{В}} = \frac{22 \cdot 12.5}{22 \cdot 0.4 + 4 \cdot 12.5 \cdot 0.85} = 5.36\ \text{Ом}.
\]

Знаходимо загальний опір заземлювача:
\[
R = \frac{R_\text{П} \cdot R_\text{Ш}}{R_\text{П} + R_\text{Ш}} = \frac{10 \cdot 5.36}{10 + 5.36} = 3.49\ \text{Ом}.
\]

Оскільки опір заземлювача менше $R_\text{Д}=4\ \text{Ом}$ розрахунок виконаний вірно.

\subsection{Висновок}
Описано шкідливі та небезпечні фактори, що діють у приміщенні, а також професійні захворювання, які можуть стати наслідками дії цих факторів.

Дано розрахунок пристрою заземлення, встановлення якого підвищує електробезпеку.

Проведено заміри основних параметрів робочого приміщення та зроблено їх аналіз відповідно до чинних норм та правил охорони праці. При цьому виявлені порушення норм та дані рекомендації щодо їх виправлення.

Виправлення виявних порушень зменшить ймовірність нещасних випадків, виробничого травматизму, професійних захворювань та збільшить продуктивність праці.


\begin{thebibliography}{9}
\bibitem{npaop1210}
	НПАОП 0.00-1.28-10
	\emph{Правила охорони праці під час експлуатації електронно-обчислювальних машин}
	
\bibitem{sanpin798}
	ДСанПіН  3.3.2-007-98
	\emph{Державні санітарні правила і норми роботи з візуальними дисплейними терміналами електронно-обчислювальних машин}
	
\bibitem{gost12}
	ГОСТ 12.0.003-74
	\emph{Шкідливі та небезпечні виробничі фактори}
	
\bibitem{snip}
	СНиП 2.09.04.-87
	\emph{Адміністративні та побутові споруди}
	
\bibitem{gost_fact}
	ГОСТ 12.0.003-74
	\emph{Небезпечні та шкідливі виробничі фактори}
	
\bibitem{burakova}
	Буракова С.О.
	\emph{Дипломне проектування. Розділи з охорони праці}
	Кам’янець-Подільський, 2010
	
\bibitem{dzunzuk}
	Дзюндзюк Б.В.
	\emph{Охорона праці. Збірник задач}
	ХНУРЕ, Харків, 2006

\end{thebibliography}

\end{document}